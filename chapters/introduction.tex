\paragraph{Games on weighted graphs.}
Weighted games are a common way to formally address questions
related to consumption, production and storage of resources: the arena
of such game are two-player turn-based games in which transitions
carry positive or negative integers, representing the accumulation or
consumption of resource.  Various objectives have been considered for
such arenas, such as optimizing the total or average amount of
resources that have been collected along the play, or maintaining the
total amount within given bounds. The~latter kind of objectives,
usually referred to as \emph{energy
objectives}~\cite{CdAHS03,BouyerFLMS08}, has been widely studied in
the untimed
setting~\cite{ChatterjeeD12,ChatterjeeDHR10,DDGRT10,FahrenbergJLS11,JLR13,
JLS15,VCDHRR15,BMRLL15,BHMRZ17,DM18}, and to a lesser extent in the
timed setting~\cite{BFLM10,qest2012-BLM}.
%
As their name indicates, energy objectives can be used to model the
evolution of the available energy in an autonomous system: besides
achieving its tasks, the~system has to take care of recharging
batteries regularly enough so as to never run out of power.  Energy
objectives were also used to model moulding machines: such machines
inject molten plastic into a mould, using pressure obtained by storing
liquid in a tank~\cite{CJLRR09}; the~level of liquid has to be
controlled in such a way that enough pressure is always available, but
excessive pressure in the tank would reduce the service life of the valve.

Energy games impose strict constraints on the total amount of energy
at all stages of the play. Two kinds of constraints have been mainly
considered in the literature: lower-bound constraints (a.k.a. L-energy constraints) impose a strict lower bound (usually~zero), but impose no
upper bound; on~the other hand, lower- and upper-bound constraints
(a.k.a. LU-energy constraints) require that the energy level always
remain within a bounded interval~$[L;U]$. Finding strategies that
realize L-energy objectives along infinite runs is in PTIME in the
one-player setting, and in NP $\cap$ coNP for two players;
for LU-energy objectives, it~is respectively PSPACE-complete
and EXPTIME-complete~\cite{BouyerFLMS08}. Some works have also considered
the existence of an initial energy level for which a winning strategy
exist~\cite{ChatterjeeDHR10}.
%
Energy objectives have also been combined with other objectives,
either qualitative (e.g.~parity~\cite{ChatterjeeD12}) or quantitative
objectives (e.g.~multi-dimensional
energy~\cite{ChatterjeeDHR10,FahrenbergJLS11,JLR13}). 

In this paper, we focus on weighted games combining energy objectives
together with reachability objectives. Our~first result is the
(expected) proof that L-energy games with or without reachability
objectives are interreducible; the same holds for LU-energy games.

%
We~then focus on relaxations of the energy constraints, in two
different directions. In both cases, the lower bound remains
unchanged, as it corresponds to running out of energy, which we always
want to avoid. We~thus only relax the upper-bound constraint.
The~first direction concerns \emph{weak upper bounds}, already
introduced in~\cite{BouyerFLMS08}: in~that setting, hitting the upper bound
is allowed, but there is no overload, i.e. trying to exceed the upper bound will simply maintain the energy level at this maximal level. Yet, a strict lower bound is
still imposed. We~name these objectives LW-energy
objectives. This~could be used as a (simplified) model for batteries.
When considered alone, LW-energy objectives are not much different
from~ L-energy objectives, in the sense that the aim is to find a
reachable \emph{positive loop}. LW-energy games are in PTIME for
one-player games, and in NP $\cap$ coNP for two players. When
combining LW-energy and reachability objectives, the~situation
changes: different loops may have different effects on the energy
level, and we have to keep track of the final energy level reached
when iterating those loops.

The second way of relaxing upper bounds, which we call \emph{soft
upper bound}, allows a limited number (or amount) of
violations: when modeling a pressure tank, the lower-bound constraint
is strict (pressure should always be available) but the upper bound is
soft (excessive pressure may be temporarily allowed is
needed). We~consider different kinds of limits (on the number or
amount of violations), and prove that (with or without reachability
objectives) the energy game problems are PSPACE-complete for one-player arenas,
and EXPTIME-complete for two-player ones.

\fbox{minimization?}


\paragraph{Related work.}
Quantitative games have been the focus of numerous research articles
 since the~1970s, with various kinds of objectives. 
%such as ultimately
% optimizing the total payoff~\cite{}, mean-payoff~\cite{EM79,ZwickP95},
% or discounted sum~\cite{ZwickP95,Andersson06}.
% %
% Energy objectives, which are a kind of safety objectives on the total
% payoff, were introduced in~\cite{CdAHS03} and rediscovered
% in~\cite{BouyerFLMS08}.
% %
% Several works have extended those works by
% combining quantitative conditions together, e.g. multi-dimensional
% energy conditions~\cite{FahrenbergJLS11,JurdzinskiLS15} or
% conjunctions of energy- and mean-payoff
% objectives~\cite{ChatterjeeDHR10}. Combinations with qualitative
% objectives (e.g. reachability~\cite{Chatterjee0H17} or parity
% objectives~\cite{ChatterjeeD12,ChatterjeeRR14}) were also considered.
% %
% Similar objectives have been considered in slightly different
% settings e.g. Vector Addition Systems with States~\cite{Reichert16} and
% one-counter machines~\cite{GHOW10,Hun15}.
% %



Other types of quantitative games consider mean payoffs. The payoff
of a run is the mean weight along this run, and the value of a
mean payoff game is the maximal/minimal mean weight that one can
enforce with an appropriate strategy. In this setting, one mainly has
to consider the mean value of cycles, that absorb all other
costs. \cite{ZwickP95} shows that the value of infinite plays can be
computed in $??$, where $p$ is the
maximal absolute value of weights in the arena.

Mean payoff games, however, only address quantitative questions as
limit values of infinite runs, and do not consider quantitative
constraints on prefixes of these runs, nor boolean objectives.

Discounted games are another form of quantitative games for which
polynomial solutions exist. In these games, a discounting
factor~$\lambda < 1$ is defined, and the contribution of the $i$-th
transition to the total payoff of a run is the weight of the
transition multiplied by $\lambda^i$. The value of these games can be
computed in PTIME using linear programming~\cite{Andersson06}.



Games with quantitative and reachability objectives have already been
considered. \cite{Chatterjee0H17} considers total payoff, discounted
payoff and energy payoff combined with a reachability objective. Given
a game graph $G$ an initial state~$s$, and a target state~$t$, the
question addressed is whether there exists a~run~$\rho$ from $s$
to~$t$ in $G$ such that the total payoff, discounted payoff, or the
energy level of $\rho$ is non negative. Checking that a run with
payoff $\geq 0$ exists for a single player, is~in~ PTIME for all payoff functions, and for
two-player games in NP $\cap$ coNP. For equality, the problems are
respectively NP-complete and EXPSPACE-complete for total and
energy payoff, and the question is still open for discounted
games. The~PTIME algorithms rely on the computation of values in
games, which can be efficiently done as soon as one does not impose
constraints on payoff achieved by prefixes of winning runs.

\cite{CdAHS03} combines energy objectives and B\"uchi objectives in
two players games; the objective is that a pair of components sharing
resources never consume more than a certain threshold, while
preserving some liveness properties, i.e. visiting some states
infinitely often.

\cite{ChatterjeeD12} combines quantitative and boolean objectives in
Energy-Parity Games: the objective is to win a parity game while
maintaining an energy constraint, i.e. ensure that the total payoff of
every prefix of a winning run remains $\geq 0$. Results are the following: first player 1 has winning strategies with memory of
size $n\cdot d \cdot W$, where $n$ is the number of states, $W$ the
largest weight, and $d$ the number of priorities in the parity
condition. Second, memoryless strategies are sufficient for player 2. Last
memoryless winning strategies exist for player 2 if the boolean condition
is a coB\"uchi condition.

\cite{ChatterjeeDHR10} generalizes mean payoff and energy games, by considering vectors of payoff functions.
These games are determined when considering finite-memory
strategies. In this setting, energy and mean payoff games are
interreducible, and threshold questions for mean payoff, or unknown
initial credit (whether there exist initial values for which an energy
game is winning) are coNP-complete.

\cite{ChatterjeeRR14} considers multiple quantitative mean payoff and
energy objectives
combined with parity objectives.  Strategies for such quantitative
games have to fulfill two objectives: first progress towards a target
state or enforce a parity condition, and second, satisfy the
constraint on a vector of quantitative outcomes of the runs: all mean
payoff are greater than a threshold $\nu$, or all energy levels are
positive. Usually these games require infinite memory. However, when
restricting to finite memory strategies, the required memory is
exponential.

\cite{BouyerFLMS08} considers energy games with strong lower bound
constraints (L), strong lower bound and strong upper bound (LU) and strong
lower bound and weak upper bound (LW) constraints. Winning runs in these
games are infinite runs which prefixes satisfy the L, LU or LW energy
constraint. For the one player setting, existence of strategies for L,
and LW games are in PTIME, and LU games are PSPACE-complete. In the two-player setting L, and LW games are in
NP $\cap$ coNP, and LU games are EXPTIME-complete.

\cite{FahrenbergJLS11} considers energy games in multi-weighted automata,
and L, LW, LU games similar to those of \cite{BouyerFLMS08} but with
vectors of energy levels and constraints attached to each energy level
in these vectors. The complexity of these games depend on the size of
energy vectors.

\cite{HunterR14} considers multiple quantitative objectives,
i.e. addresses the problem of determining if a player can attain a
payoff in a finite union of arbitrary intervals for various payoff
functions (liminf, mean-payoff, discount sum, total~sum). Given a
fixed number of intervals $I_1, \dots I_k$ and a payoff function~$f$,
a play~$\pi$ is winning if $f(\pi) \in I_j$ for some $j\in [1;k]$. The
complexity of these games depend on the payoff function and ranges from
NP $\cap$ coNP to EXPSPACE.


\cite{JurdzinskiLS15} considers games in multi-weighted automata and
the initial credit problem, i.e. whether, starting from a vertex~$v$,
there exists a vector~$B\in \mathbb N^k$ of initial weights such that player 1 has a
winning strategy starting from configuration $(v,B)$. They show that
this problem is 2EXPTIME-complete for general multi-dimensional energy
games.

\cite{Reichert16} Considers reachability in counter games under different semantics. Configurations of the game are counter values, and moves are labeled by integer vectors. The semantics of the arena either imposes no constraint on legal values of counters ($\mathbb{ Z}$ semantics), forbids moves that would result in a negative value of some counter (blocking VASS semantics), or considers VASS semantics with weak lower bounds (all moves are legal, but counter values have $0$ lower bound). A play is winning if it reaches a configuration from a target set. In dimension 2, under every semantics, two-player reachability games are undecidable. In dimension 1, these games are in EXPSPACE. 




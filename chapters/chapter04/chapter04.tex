In this chapter, we will consider the quantitative games, where player $1$ has to reach his goal, always maintaining his weights inside two bounds. This chapter we will consider both the bounds to be strong.

\section{Finite Total Payoff Reachability}

In this section, we will use total payoff as our quantitative function. Given a game graph $G=\langle Q_1, Q_2, E, w, q_0, T \rangle$, an upper bound $U \in \mathbb{N}$ and a lower bound $b \in \mathbb{N}$, finite total payoff reachability objective with dual bound says that, with zero initial energy starting from $q_0$, player $1$ has to reach $T$ in a path $\rho$ such that, for all finite prefixes $\rho^{\prime}$ of $\rho$, $l\leq TP(\rho^{\prime})\leq U$, where $TP$ is simply the sum of the weight of the edges as defined earlier. We call this \textbf{LU -Reachability Game}. We will consider, both one player and two player version of the game. 
\subsection{One Player LU-Reachability Game}
We will consider the case where $Q_2= \phi$. Hence, all the vertices are player $1$ vertices. We will prove the following theorem:\\
\begin{theorem}
\label{pspace-complete}
One player LU-reachability game is PSPACE-complete.
\end{theorem}
\begin{proof}
\end{proof}

\subsection{Two player LU Game}
Now we will move to the case of two player LU-reachability Game. Here $Q_2 \not = \phi$ anymore. We will prove the following theorem:
\begin{theorem}
\label{exp-complete}
Two players LU-reachability game is EXPTIME-complete
\end{theorem}
\begin{proof}

\end{proof}

\se
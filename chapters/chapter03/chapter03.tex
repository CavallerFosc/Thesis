In this chapter, we will see different quantitative games with single bound. \\

\section{Energy-Reachability Game}
Given a game graph $G=\langle Q_1, Q_2, E, w, q_0, T \rangle$, with all usual notations, the energy-reachability objective is to reach the target states $T$ starting from $q_0$ in a path $\rho$, such that for all finite prefixes $\pi(\rho)$, $w(\pi) \geq 0$.\\

Now, given a game graph, the decision problem for this game is that if Player $1$ can win maintaining energy-reachability objective? We will prove the following theorem:\\
\begin{theorem}
\label{energy-reach-thm}
The decision problem for Energy-Reachability game is in $NP \cap coNP$. For the lower bound, the problem is mean-payoff hard.
\end{theorem}
\begin{proof}
We will prove that, energy-reachability game is equivalent to infinite-energy game by both side reduction. The theorem will be the consequence of the proof from ~\cite{DBLP:conf/formats/BouyerFLMS08}.\\

Given a game graph $G=\langle Q_1, Q_2, E, w, q_0\rangle$, with all usual notations, the decision problem for infinite-energy game asks, if player $1$ has a strategy such that with zero initial weight for any infinite play $\gamma$ from initial state $q_0$, if $w(\gamma^{\prime}) \geq 0$ for all finite prefixes $\gamma^{\prime}$ of $\gamma$? \\
First we reduce the infinite-energy game to energy-reachability game.
Consider a game graph $G=\langle Q_1, Q_2, E, w, q_0\rangle$. We produce another game graph $G^{\prime}=\langle Q_1, Q_2, E^{\prime}, w^{\prime}, q_0,T\rangle$, where we add a target vertex $T$. For every player $1$ vertex $v \in Q_1$, we add an edge $(v,T) \in E^{\prime}$ with $w^{\prime}(v,T) = -\delta$, where $\delta$ is larger than the sum of all the positive weights of the graph $G$. For, all other edges $e \in E$, we have the same corresponding edges $e^{\prime} \in E^{\prime}$ with $w^{\prime}(e^{\prime})= w(e)+ \epsilon$, where $\epsilon << 1/2n$, where $n$ is the total number of vertices. The overview of the construction is shown in Figure \ref{inf-reach}.
\begin{figure}
    \label{inf-reach}
    \centering
    \tikzset{every picture/.style={line width=0.75pt}} %set default line width to 0.75pt        

\begin{tikzpicture}[x=0.75pt,y=0.75pt,yscale=-1,xscale=1]
%uncomment if require: \path (0,300); %set diagram left start at 0, and has height of 300

%Shape: Ellipse [id:dp35665433870720253] 
\draw   (28,115) .. controls (28,81.31) and (74.23,54) .. (131.25,54) .. controls (188.27,54) and (234.5,81.31) .. (234.5,115) .. controls (234.5,148.69) and (188.27,176) .. (131.25,176) .. controls (74.23,176) and (28,148.69) .. (28,115) -- cycle ;
%Shape: Circle [id:dp8502084996926933] 
\draw   (59,96.5) .. controls (59,89.04) and (65.04,83) .. (72.5,83) .. controls (79.96,83) and (86,89.04) .. (86,96.5) .. controls (86,103.96) and (79.96,110) .. (72.5,110) .. controls (65.04,110) and (59,103.96) .. (59,96.5) -- cycle ;
%Shape: Square [id:dp9450962306071624] 
\draw   (137,82) -- (159,82) -- (159,104) -- (137,104) -- cycle ;
%Straight Lines [id:da2842158542061044] 
\draw    (86,96.5) -- (135.5,93.14) ;
\draw [shift={(137.5,93)}, rotate = 536.11] [color={rgb, 255:red, 0; green, 0; blue, 0 }  ][line width=0.75]    (10.93,-3.29) .. controls (6.95,-1.4) and (3.31,-0.3) .. (0,0) .. controls (3.31,0.3) and (6.95,1.4) .. (10.93,3.29)   ;

%Shape: Ellipse [id:dp9633483807744296] 
\draw   (338,110) .. controls (334.57,76.31) and (378.02,49) .. (435.04,49) .. controls (492.07,49) and (541.08,76.31) .. (544.5,110) .. controls (547.93,143.69) and (504.48,171) .. (447.46,171) .. controls (390.43,171) and (341.42,143.69) .. (338,110) -- cycle ;
%Shape: Circle [id:dp3748033424118844] 
\draw   (367.12,91.5) .. controls (366.36,84.04) and (371.79,78) .. (379.24,78) .. controls (386.7,78) and (393.36,84.04) .. (394.12,91.5) .. controls (394.88,98.96) and (389.45,105) .. (381.99,105) .. controls (374.53,105) and (367.87,98.96) .. (367.12,91.5) -- cycle ;
%Shape: Square [id:dp9047184358152496] 
\draw   (443.64,77) -- (465.64,77) -- (467.88,99) -- (445.88,99) -- cycle ;
%Straight Lines [id:da04429014923239061] 
\draw    (394.12,91.5) -- (443.26,88.14) ;
\draw [shift={(445.26,88)}, rotate = 536.0899999999999] [color={rgb, 255:red, 0; green, 0; blue, 0 }  ][line width=0.75]    (10.93,-3.29) .. controls (6.95,-1.4) and (3.31,-0.3) .. (0,0) .. controls (3.31,0.3) and (6.95,1.4) .. (10.93,3.29)   ;
%Shape: Circle [id:dp6442851353260122] 
\draw   (428,227) .. controls (428,213.19) and (439.19,202) .. (453,202) .. controls (466.81,202) and (478,213.19) .. (478,227) .. controls (478,240.81) and (466.81,252) .. (453,252) .. controls (439.19,252) and (428,240.81) .. (428,227) -- cycle ;
%Straight Lines [id:da29377848783032556] 
\draw    (381.99,105) -- (440.49,205.27) ;
\draw [shift={(441.5,207)}, rotate = 239.74] [color={rgb, 255:red, 0; green, 0; blue, 0 }  ][line width=0.75]    (10.93,-3.29) .. controls (6.95,-1.4) and (3.31,-0.3) .. (0,0) .. controls (3.31,0.3) and (6.95,1.4) .. (10.93,3.29)   ;


% Text Node
\draw (107,83) node  [align=left] {w};
% Text Node
\draw (413.74,78) node  [align=left] {w + $\displaystyle \epsilon $};
% Text Node
\draw (453,227) node  [align=left] {T};
% Text Node
\draw (397,179) node  [align=left] {
\begin{itemize}
\item \mbox{-}$\displaystyle \delta $
\end{itemize}};
% Text Node
\draw (117,27) node [xslant=0.06] [align=left] {G};
% Text Node
\draw (436,18) node  [align=left] {G$\displaystyle ^{\prime }$};


\end{tikzpicture}

    \caption{Infinite-energy to Energy-reachability}
\end{figure}

If player $1$ has a strategy to win infinite-energy in $G$, then all the cycles player $1$ forces in the infinite winning path $\rho$ is either a positive or a zero cycle. The same path in $G^{\prime}$ will produce positive cycles for $\epsilon$. Hence, after a large number of iterations, player $1$, will have weight $\geq \delta$ in some of its vertex and then it can reach $T$ in $G^{\prime}$ and win energy-reachability. For the other side, let player $1$ has a winning strategy in energy-reachability in $G^{\prime}$. That means, in every winning path, at some player $1$ vertex it has weight $\geq \delta$, which is impossible without forcing a positive cycle as $\delta$ is large enough. Hence, iterating the same cycle player $1$ wins infinite-energy game in $G$.\\
Now, we will show the opposite reduction i.e. from energy-reachability to infinite-energy. Again consider, a game graph $G=\langle Q_1, Q_2, E, w, q_0, T\rangle$. We produce another game graph $G^{\prime}=\langle {Q_1}^{\prime}, {Q_2}^{\prime}, E^{\prime}, w^{\prime}, q_0\rangle$, where we add a self loop of weight $0$ on $T$ and we keep only those vertices and corresponding edges in $G^{\prime}$ which are in the attractor set of $T$ in $G$. Clearly, if player $1$ wins energy-reachability in $G$, he wins infinite-energy in $G^{\prime}$ by taking the zero loop in $T$ after reaching there. The other direction is also true as if player $1$ wins infinite energy in $G^{\prime}$, either he takes the loop in $T$ or he goes into a positive loop inside. If he reaches $T$, he wins energy-reachability in $G$ trivially. If he can loop inside, as the graph is restricted to attractor set of $T$, he can force to reach $T$ and win energy-reachability in $G$. Hence, it is proved that, both the games are equivalent complexity -wise.
\end{proof}

\section{Finite Mean-Payoff Reachability}
In this section, we will take the mean-payoff function as our quantitative function. In general mean-payoff is used in theory mostly for the case of infinite games, but in this thesis we restrict ourselves to reachability, hence finite games. For this reason, we will define a notion of mean payoff for the finite case:\\
\vskip 0.5cm
Given a game graph $G=\langle Q_1, Q_2, E, w, q_0,T\rangle$ and a bound $b \in \mathbb{N}$, the decision problem for finite mean-payoff reachability game asks, starting from $q_0$ with zero initial weight, if player $1$ has a strategy to reach $T$ in some path $\gamma$ such that, for all finite prefixes $\gamma^{\prime}$ of $\gamma$, if $MP(\gamma^{\prime}) \lesseqgtr b$? We will only consider the case of  $MP(\gamma^{\prime}) \leq b$ here. The other case is just the dual. Remember that, for a prefix $\gamma^{\prime}= q_0,q_1,\ldots q_l$, $MP(\gamma^{\prime})= (1/l) \cdot \Sigma_{i=0}^{l-1} w(q_i,q_i+1)$.\\ Now, we will prove the following result about the finite mean-payoff reachability game:\\
\begin{theorem}
\label{fin-meanpayoff-thm}
The decision problem for finite mean-payoff reachability game is in $NP \cap coNP$ and also mean-payoff hard. 
\end{theorem}
\begin{proof}
We will reduce this game to and from energy-reachability game. As usual, we consider a game graph\\ $G=\langle Q_1, Q_2, E, w, q_0,T\rangle$. We produce another game graph, $G^{\prime}=\langle Q_1, Q_2, E, w^{\prime}= b-w (\text{or } w-b, \text{accordingly}), q_0,T\rangle$. We claim that, player $1$ wins energy-reachability in $G$ iff he wins finite mean-payoff reachability in $G^{\prime}$ and also vice-versa. This will prove that, energy-reachability is equivalent to mean-payoff reachability complexity-wise. The result of the theorem will be then the consequence of Theorem \ref{energy-reach-thm}\\
First we prove the reduction from energy reachability to mean-payoff reachability. The opposite reduction will be exactly similar. Let player $1$ has a winning strategy $\lambda$ in $G$ for energy reachability. That means, for all the path $\rho$ conforming $\lambda$, reaches $T$ and for all finite prefixes $\rho^{\prime}$ of $\rho$, $w(\rho)\geq 0$. Consider such a prefix $q_0,q_1,\ldots q_l$. We have, $\Sigma_{i=0}^{l-1} w(q_i,q_{i+1}) \geq 0$. Now, let in $G^{\prime}$, also player $1$ plays with same strategy $\lambda$. Now, for the same prefix, mean-payoff will be:
\begin{align}
\notag
\centering
MP(\rho^{\prime})&= (1/l)\Sigma_{i=0}^{l-1} w^{\prime}(q_i,q_{i+1})\\
&=(1/l)\Sigma_{i=0}^{l-1} (b- w(q_i,q_{i+1}))\\
&=(1/l)(lb- \Sigma_{i=0}^{l-1} w(q_i,q_{i+1})\\
&=b- (1/l)\Sigma_{i=0}^{l-1} w(q_i,q_{i+1})\\
&\leq b
\end{align}


Hence, player $1$ wins in finite mean-payoff reachability objective in $G^{\prime}$. The other proofs can be obtained similarly.\\
Hence, we prove that, finite mean-payoff reachability is equivalent to energy-reachability.

\end{proof}

\section{Finite Total Payoff Reachability}
Now, it comes down to the final quantitative function we are concerned with in this thesis, which is total payoff. Given a game graph $G=\langle Q_1, Q_2, E, w, q_0,T\rangle$ and a bound $b \in \mathbb{N}$, the decision problem for finite total payoff reachability game asks, starting from $q_0$ with zero initial weight, if player $1$ has a strategy to reach $T$ in some path $\gamma$ such that, for all finite prefixes $\gamma^{\prime}$ of $\gamma$, if $TP(\gamma^{\prime}) \lesseqgtr b$? We will only consider the case of  $TP(\gamma^{\prime}) \geq b$ here. The other case is just the dual. Remember that, for a prefix $\gamma^{\prime}= q_0,q_1,\ldots q_l$, $TP(\gamma^{\prime})= \Sigma_{i=0}^{l-1} w(q_i,q_i+1)$. Note that, energy-reachability is just a special case of total payoff reachability, where $b=0$. We will show that, it is equivalent to energy-reachability game. From that, the following theorem is evident:\\
\begin{theorem}
\label{fin-totalpayoff-thm}
The decision problem for finite total payoff reachability game is in $NP \cap coNP$ and also mean-payoff hard. 
\end{theorem}
\begin{proof}
\huge\fcolorbox{blue}{red}{Incomplete}
\end{proof}

\section{Conclusion}
This brings us to the end of the Quantitative Reachability Games with Single Bounds. We showed that, all the reachability games with single bounds (for the quantitative functions we are concerned with in this thesis) are equivalent to energy-reachability game which is complexity-wise equivalent to infinite energy games. So, we have $NP \cap coNP$ complexity for all of them. For the lower bound also, all of them are at least mean-payoff hard. Next we move to the chapters for the similar kind of games with two bounds. 